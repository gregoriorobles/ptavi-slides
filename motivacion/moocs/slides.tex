%
% $Id: $
%
%
% Compilar a .pdf con LaTeX (pdflatex)
% Es necesario instalar Beamer (paquete latex-beamer en Debian)
%

%
% Gr�ficos:
% Los gr�ficos pueden suministrarse en PNG, JPG, TIF, PDF, MPS
% Los EPS deben convertirse a PDF (usar epstopdf)
%

\documentclass{beamer}
\usetheme{JuanLesPins}
%\usebackgroundtemplate{\includegraphics[width=\paperwidth]{format/libresoft-bg.png}}
%\usepackage[spanish]{babel}
\usepackage[latin1]{inputenc}
\usepackage{graphics}
\usepackage{amssymb} % Simbolos matematicos

\ProcessOptions

%% Metadatos del PDF.
\hypersetup{
  pdftitle={Protocolos para la Transmisi�n de Audio y V�deo por Internet},
  pdfauthor={Gregorio Robles, Jes�s M. Gonz�lez Barahona},
  pdfcreator={GSyC, Universidad Rey Juan Carlos},
  pdfproducer=PDFLaTeX,
  pdfsubject={Protocolos para la Transmisi�n de Audio y V�deo por Internet},
}
%%

\begin{document}

\title{MOOCs: Massive On-line Open Courses}
\subtitle{Protocolos para la Transmisi�n de Audio y V�deo en Internet}
\institute{\{grex,jgb\}@gsyc.urjc.es \\
GSyC, Universidad Rey Juan Carlos}
\author[Gregorio Robles, Jes�s M. Gonz�lez Barahona]{Gregorio Robles, Jes�s M. Gonz�lez Barahona}
\date[Oct 2016]{19 de octubre de 2016}


\frame{
\maketitle
}


% Si el titulo o el autor se quieren acortar para los pies de p�gina
% se pueden redefinir aqu�:
%\title{Titulo corto}
%\author{Autores abreviado}


%% LICENCIA DE REDISTRIBUCION DE LAS TRANSPAS
\frame{
~
\vspace{4cm}

\begin{flushright}
\includegraphics[width=2.2cm]{figs/by-sa}

{\tiny
(cc) 2014 Gregorio Robles, Jes�s M. Gonz�lez Barahona \\
  Some rights reserved. This work licensed under Creative Commons \\
  Attribution-ShareAlike License. To view a copy of full license, see \\
\ \\
  http://creativecommons.org/licenses/by-sa/3.0/ or write to \\
  Creative Commons, 559 Nathan Abbott Way, Stanford, \\
  California 94305, USA. \\
\ 
}
\end{flushright}
}
%%

%-----------------------    ---------------------------------

\begin{frame}
\frametitle{�Qu� son los MOOCs?}

\begin{itemize}
   \item Cursos por Internet
   \item Hay algunos muy buenos, generalmente en ingl�s
   \item Generalmente gratis (algunos cobran por certificado, si lo terminas)
   \item Muchos de ellos ofrecidos por instituciones de renombre
   \item Basados generalmente en v�deos, lecturas y entrega de ejercicios
   \item Hay de todo: tecnol�gicos, de econom�a, de programaci�n...
\end{itemize}

\end{frame}


%-----------------------    ---------------------------------

\begin{frame}
\frametitle{Sitios de MOOCs}

\begin{center}
  \includegraphics[width=10cm]{figs/sitios.jpg}
\end{center}


\begin{flushright}
{\tiny
Source: http://www.vocal.ie/wp-content/uploads/2014/06/MOOCs-Daigram11.jpg
}
\end{flushright}

\end{frame}

%-----------------------    ---------------------------------

\begin{frame}
\frametitle{Plataformas recomendadas}

\begin{itemize}
   \item Coursera (existe la aplicaci�n CourseraCast para ver los v�deos con el Chromecast en la TV)
   \item edX: del MIT
   \item Udacity: spin-off de Univ. Stanford
   \item Mir�adaX (en espa�ol)
\end{itemize}

\end{frame}



\frame{
\maketitle
}

\end{document}
